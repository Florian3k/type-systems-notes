\chapter{Data Abstraction}

%%%%%%%%%%%%%%%%%%%%%%%%%%%%%%%%%%%%%%%%%%%%%%%%%%%%%%%%%%%%%%%%%%%%%%%%%%%%%%%
\section{Existential Types}

\newcommand\Int[0]{\texttt{int}}
\newcommand\Pack[1]{\texttt{pack}\;#1}
\newcommand\Unpack[3]{\texttt{unpack}\;#1 = #2\;\texttt{in}\;#3}

One of key features of modern computer languages, commonly associated with
Object Oriented Programming, is the ability to abstract implementation details
of data structures out, limiting the set of operations that a user can perform
directly on this structure to those defined as an interface.
In C++ this behaviour is implemented by classes while in OCaml it is achieved
by using modules, for example:
\begin{verbatim}
    module M : sig
        type t
        val f : int -> t
        val g : t -> int
    end
\end{verbatim}

The main idea is to hide the actual type of the structure, that is to substitute
it with a new, abstract type, about which no information is available,
what forces the user to use only the methods we created. The type system,
after the structure is defined, should only remember there exists a type, but
forget all associations with the actual type. That is why we call such types
\textit{existential types}.

As usual, we augment our grammar by syntax of the types and two new operations:
one for introduction of the type and one for its elimination.
\begin{alignat*}{2}
  \tau & \Coloneqq \ldots \mid \exists\alpha.\tau \\
  e    & \Coloneqq \ldots \mid \Pack{e} \mid \Unpack{x}{e}{e} \\
  v    & \Coloneqq \ldots \mid \texttt{pack}\;v
\end{alignat*}

$\tau$ should be the type of the interface (i.e. of functions operating on
$\alpha$). The module from above would be translated to
$\exists \alpha.(\Int \to \alpha) \times (\alpha \to \Int)$.

$\texttt{pack}$ creates the object of an abstract type and $\texttt{unpack}$
opens it, making methods available (but it does not disclose the type $\alpha$).
\begin{mathpar}
  \inferrule{\Delta; \Gamma \vdash e \colon \tau\{\tau'/\alpha\}}
            {\Delta; \Gamma \vdash \Pack{e} \colon \exists\alpha.\tau}

  \inferrule{\Delta; \Gamma \vdash e_1 \colon \exists\alpha.\tau \\
             \Delta, \alpha; \Gamma, x \colon \tau \vdash e_2 : \tau'}
            {\Delta; \Gamma \vdash \Unpack{x}{e_1}{e_2} : \tau'}
\end{mathpar}

We have a standard reduction rule for unpacking a value.
\begin{mathpar}
  \inferrule{ }
            {\Unpack{x}{\Pack{v}}{e} \rightharpoonup e\{v/x\}}
\end{mathpar}

Abstract interface for pairs (constructor, left selector, right selector):

$\exists\alpha.(\tau_1 \to \tau_2 \to \alpha) \times (\alpha \to \tau_1) \times (\alpha \to \tau_2)$

Implementation using "regular" pairs:

$\Pack{(\lambda x \ldotp \lambda y \ldotp (x,y), \lambda p \ldotp \pi_1 p, \lambda p \ldotp \pi_2 p)}$

Implementation using church encoding:

$\Pack{(\Lambda \lambda x \ldotp \lambda y \ldotp \lambda f \ldotp f\;x\;y,
        \lambda p \ldotp p *(\lambda x \ldotp \lambda y \ldotp x),
        \lambda p \ldotp p *(\lambda x \ldotp \lambda y \ldotp y))}$

%%%%%%%%%%%%%%%%%%%%%%%%%%%%%%%%%%%%%%%%%%%%%%%%%%%%%%%%%%%%%%%%%%%%%%%%%%%%%%%
\section{Contextual Equivalence}
\label{sec:ctx-equiv}

General closing contexts

\begin{alignat*}{2}
  C & \Coloneqq \square \mid C\;e \mid e\;C \mid \lambda x.C \mid \ldots
\end{alignat*}

We define a relation for typing closing contexts:
\[
  C \colon (\Delta; \Gamma \vdash \tau) \leadsto \tau'
    \quad\Longleftrightarrow\quad
    \forall e\ldotp (\Delta; \Gamma \vdash e \;:\;\tau) \Rightarrow
      \varnothing;\varnothing \vdash C[e] \colon \tau'
\]

Alternatively, typing of closing contexts can be defined by
a set of inference rules.

Similarly to definition of $\Obs$ in \autoref{sec:biorthogonal-closure}:

\begin{defin}[Observational Equivalence]
  Programs $p_1$ and $p_2$ are \emph{observationally equivalent}
  (written $p_1 \simeq_{obs} p_2$)
  iff $p_1 \Downarrow \iff p_2 \Downarrow$.
\end{defin}

\begin{defin}[Contextual Equivalence]
  Expressions $e_1$ and $e_2$ are \emph{contextually equivalent}
  (written $\Delta; \Gamma \models e_1 \simeq_{ctx} e_2$)
  iff for any closing context $C$ such that $C : (\Delta;\Gamma\vdash\tau) \leadsto \tau'$
  for some $\tau'$
  we have $C[e_1] \simeq_{obs} C[e_2]$.
\end{defin}

%%%%%%%%%%%%%%%%%%%%%%%%%%%%%%%%%%%%%%%%%%%%%%%%%%%%%%%%%%%%%%%%%%%%%%%%%%%%%%%
\section{Binary Logical Relations}
\[
  \semcatset{Type} = \mathcal{P}(\syncatset{Value}_{\varnothing}^2),
\]

\[
  \Denot{\cdot} \colon \syncatset{Type} \to \semcatset{Type}
\]

\begin{alignat*}{2}
  \Denot{\texttt{Unit}}_{\eta} & = \{ (\texttt{()}, \texttt{()}) \} \\
  \Denot{\tau_1 \to \tau_2}_{\eta} & = \{ (v_1, v_2) \mid
    \forall (v_1', v_2') \in \Denot{\tau_1}_{\eta}\ldotp
    (v_1\;v_1', v_2\;v_2' ) \in \mathcal{E}\Denot{\tau_2}_{\eta} \} \\
  \Denot{\forall\alpha.\tau}_{\eta} & = \{ (v_1, v_2) \mid \forall R \in \semcatset{Type}.
    (v_1\;*, v_2\;*) \in \mathcal{E}\Denot{\tau}_{\eta[\alpha \mapsto R]} \} \\
  \Denot{\alpha}_{\eta} & = \eta(\alpha) \\
  \Denot{\exists\alpha.\tau}_{\eta} & =
    \{ (\texttt{pack}\;v_1,\texttt{pack}\;v_2) \mid \exists R \in \semcatset{Type}.
    (v_1, v_2) \in \mathcal{E}\Denot{\tau}_{\eta[\alpha \mapsto R]} \} \\\\
  \mathcal{E}R & = \{ (e_1, e_2) \mid \forall (E_1, E_2) \in \mathcal{K} R. E_1[e_1] \simeq_{obs}  E_2[e_2]\}\\
  \mathcal{K}R & = \{ (E_1, E_2) \mid \forall (v_1, v_2) \in R. E_1[v_1] \simeq_{obs}  E_2[v_2]\}\\
\end{alignat*}


\begin{defin}[Logical Equivalence]
  Expressions $e_1$ and $e_2$ are \emph{logically equivalent}
  (written $\Delta; \Gamma \models e_1 \simeq_{log} e_2$)
  iff
  \[
    \forall \eta \colon \Delta \to \semcatset{Type}\ldotp
    \forall (\gamma_1, \gamma_2) \in \Denot{\Gamma}_{\eta} \\
    (\gamma_1^*e_1, \gamma_2^*e_2) \in \mathcal{E}\Denot{\tau}_{\eta}
  \].
\end{defin}

\begin{theorem}[Fundamental Property]
  If $\Delta; \Gamma \vdash e \;:\; \tau$ then $\Delta; \Gamma \models e \simeq_{log} e \;:\; \tau$.
\end{theorem}

\begin{theorem}[Soundness]\label{thm:soundness-bin-log-rel}
  If $\Delta; \Gamma \models e_1 \simeq_{log} e_2 \;:\; \tau$
  then $\Delta; \Gamma \models e_1 \simeq_{ctx} e_2 \;:\; \tau$.
  This can be simply written as $\simeq_{log} \subseteq \simeq_{ctx}$.
\end{theorem}

\begin{lemma}[Compatibility for $\lambda$]
  If $\Delta; \Gamma, x : \tau_1 \models e_1 \simeq_{log} e_2 \;:\; \tau_2$ then
  $\Delta; \Gamma \models \lambda x.e_1 \simeq_{log} \lambda x.e_2 \;:\; \tau_1 \to \tau_2$.
\end{lemma}

%%%%%%%%%%%%%%%%%%%%%%%%%%%%%%%%%%%%%%%%%%%%%%%%%%%%%%%%%%%%%%%%%%%%%%%%%%%%%%%
\section{Parametricity and Free Theorems}

Throughout learning functional programming there is this recurring statement
that there exists only one function of type
$\forall\alpha\ldotp\alpha\to\alpha$,
namely identity ($\id = \Lambda\lambda x\ldotp x$).
To make this statement actually true, we need to make some restrictions about
out programming language. First of all it needs to be \emph{pure} meaning no
side effects, because we could write many functions of type
$\forall\alpha\ldotp\alpha\to\alpha$ like
$\lambda x\ldotp\texttt{print 'a'};\;x$.
We also need to restrict out language from operator such as \texttt{call/cc}.
Another aspect to consider is recursion. When we introduce it there are at
least three different values of type $\forall\alpha\ldotp\alpha\to\alpha$:
\begin{itemize}
  \item $\id$
  \item $\Lambda \lambda x\ldotp\Omega$
  \item $\Lambda \ldotp\Omega$.
\end{itemize}
Where $\Omega$ is any non-terminating expression, which we should be able
to produce using recursion. The exact derivation depends on typing rules.
Those constraints are nearly exactly what we need to make our original
statement true. There is only one snag. Expression $\lambda x\ldotp \id\; x$
is not syntactically the same as $\id$. However we intuitively consider those
expressions to be the same. To formally define it we can use
contextual equivalence from \autoref{sec:ctx-equiv}. In sum, our original
statement should state that in a pure System F without recursion with
respect to contextual equivalence there exists only one value of type
$\forall\alpha\ldotp\alpha\to\alpha$. We will prove this in a moment,
but first in order to make this proof possible we need to revise our
definition of expression closure. One option would be to define it as:
  \[
    \mathcal E R= \{ (e_1, e_2) \mid \exists (v_1, v_2) \in R\ldotp
      e_1 \longrightarrow^* v_1 \wedge e_2 \longrightarrow^* v_2 \}.
  \]

Before we get to the actual proof, we will show following lemma:
\begin{lemma}\label{lem:identity}
  For any values $v,$ $v'$ if $v:\; \forall\alpha\ldotp \alpha \to \alpha$,
  then there exists value $v''$ such that $v\;* \longrightarrow^* v''$ and
  $v'' \; v' \longrightarrow^* v'$.
\end{lemma}
\begin{proof}
  Our assumption states that
    $(v, v) \in \Denot{\forall\alpha\ldotp\alpha\to\alpha}_\eta$.\\
  Unfolding the definition of denotation we get\\
  $\forall R\in\semcatset{Type}\ldotp (v\;*, v\;*) \in
    \mathcal E \Denot{\alpha \to \alpha}_{\eta[\alpha\mapsto R]}$.\\
  Since we can choose $R$, we will define it as $R = \{(v', v')\}$.\\
  Following the definition of expression closure we take\\
  $(v_1, v_2) \in \Denot{\alpha \to \alpha}_{\eta[\alpha \mapsto R]}$
    such that $v\;* \longrightarrow^* v_1$.\\
  Unfolding definition of denotation we get
    $\forall(v_1', v_2')\in R\ldotp (v_1\; v_1', v_2\; v_2')\in\mathcal E R$.\\
  Setting $v_1' = v_2' = v'$ we get $v_1 \;v' \longrightarrow^* u$.
    Since $R$ contains only one pair $(v',v')$,\\
  we get $u=v'$ and
    $v\;*\; v'\longrightarrow^* v_1\; v' \longrightarrow^* v'$.
\end{proof}

\begin{theorem}
  If $v:\; \forall\alpha\ldotp\alpha\to\alpha$, then
  $v$ and $\Lambda\lambda x\ldotp x$ are contextually equivalent.
\end{theorem}
\begin{proof}
  By \autoref{thm:soundness-bin-log-rel}
    if suffices to show logical equivalence.\\
  To show: $(v, \Lambda\lambda x\ldotp x) \in
    \Denot{\forall\alpha\ldotp\alpha\to\alpha}_\eta$.\\
  Take any $R \in \semcatset{Type}$;
    to show: $(v\; *, \Lambda\lambda x\ldotp x\;*) \in
      \mathcal E\Denot{\alpha\to\alpha}_{\eta[\alpha\mapsto R]}$.\\
  By \autoref{lem:identity} and
    $(\Lambda\lambda x\ldotp x)\;* \longrightarrow \lambda x\ldotp x$
      it suffices to show \\
  $(v'', \lambda x\ldotp x )\in
    \Denot{\alpha\to\alpha}_{\eta[\alpha \mapsto R]}$.\\
  Take any $(v_1, v_2) \in R$;
    to show: $(v''\; v_1, (\lambda x\ldotp x) v_2) \in \mathcal E R$.\\
  By assumption about $v''$ and properties of reduction it suffices to show:\\
  $(v_1, v_2) \in \mathcal E R$,
    which follows from assumption $(v_1, v_2) \in R$.
\end{proof}

%%%%%%%%%%%%%%%%%%%%%%%%%%%%%%%%%%%%%%%%%%%%%%%%%%%%%%%%%%%%%%%%%%%%%%%%%%%%%%%
\section{Further Reading}

