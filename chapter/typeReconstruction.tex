\chapter{Type Reconstruction}
\label{ch:type-reconstruction}

\newcommand\UVarX{?\!X}
\newcommand\UVarY{?\!Y}
\newcommand\UVarZ{?\!Z}
\newcommand\UVarR{?\!R}
\newcommand\unify{\stackrel{?}{=}}
\newcommand\FUVars[1]{\mathrm{fuv}(#1)}

%%%%%%%%%%%%%%%%%%%%%%%%%%%%%%%%%%%%%%%%%%%%%%%%%%%%%%%%%%%%%%%%%%%%%%%%%%%%%%%
\section{Simple Types}

Syntax of types with unification variables:
$\tau \Coloneqq \tau \to \tau \mid \syncatset{Bool} \mid \UVarX$

Inference rules:

\begin{mathpar}
  \inferrule{(x : \tau) \in \Gamma}
            {\Gamma\vdash x \Uparrow \tau / \emptyset}

  \inferrule{\Gamma, x : \UVarX \vdash e \Uparrow \tau_2 / \theta}
            {\Gamma\vdash \lambda x.e \Uparrow \theta^*(\UVarX) \to \tau_2 / \theta}

  \inferrule{\Gamma\vdash e_1 \Uparrow \tau_2 \to \tau_1 / \theta_1 \\ \theta_1^* \Gamma\vdash e_2 \Downarrow \tau_2 / \theta_2}
            {\Gamma\vdash e_1\;e_2 \Uparrow \tau_1 / \theta_2 \circ \theta_1}

  \\\\

  \inferrule{\Gamma\vdash e_1 \Uparrow \UVarX / \theta_1 \\ \theta_1'^*\Gamma\vdash e_2 \Downarrow \UVarY / \theta_2}
            {\Gamma\vdash e_1\;e_2 \Uparrow \theta_2^*(\UVarZ) / \theta_2 \circ \theta_1'}

  \text{where }\theta_1' = \theta_1[\UVarX \mapsto \UVarY \to \UVarZ]

  \inferrule{\Gamma\vdash e \Uparrow \tau' / \theta \\ \theta^*\tau \unify \tau' / \theta'}
            {\Gamma\vdash e \Downarrow \tau / \theta' \circ \theta}
\end{mathpar}

Type unification:

\begin{mathpar}
  \inferrule{ }
            {\syncatset{Bool} \unify \syncatset{Bool} / \emptyset}

  \inferrule{\tau_1 \unify \tau_2 / \theta_1 \\ \theta_1^* \tau_1' \unify \theta_1^* \tau_2' / \theta_2}
            {\tau_1 \to \tau_1' \unify \tau_2 \to \tau_2' / \theta_2 \circ \theta_1}

  \inferrule{ }
            {\UVarX \unify \UVarX / \emptyset}

  \inferrule{\UVarX \not\in \FUVars{\tau}}
            {\UVarX \unify \tau / [\UVarX \mapsto \tau]}

  \inferrule{\UVarX \not\in \FUVars{\tau}}
            {\tau \unify \UVarX / [\UVarX \mapsto \tau]}
\end{mathpar}

%%%%%%%%%%%%%%%%%%%%%%%%%%%%%%%%%%%%%%%%%%%%%%%%%%%%%%%%%%%%%%%%%%%%%%%%%%%%%%%
\section{Imperative Representation of Terms}

\begin{verbatim}
  type tp_view =
    | Bool
    | Arr of tp * tp
    | UVar of uvar
  and uvar = tp option ref
  and tp = tp_view

  let rec view tp =
    match tp with
    | UVar x ->
      begin match !x with
      | None -> tp
      | Some tp ->
        let t = view tp in
        x := Some t;
        t
      end
    | _ -> tp

  let set_uvar u t =
    (* here we can assert that u is None *)
    u := Some t
\end{verbatim}

%%%%%%%%%%%%%%%%%%%%%%%%%%%%%%%%%%%%%%%%%%%%%%%%%%%%%%%%%%%%%%%%%%%%%%%%%%%%%%%
\section{ML-Polymorphism}

\newcommand\Let[3]{\texttt{let}\;#1 = #2\;\texttt{in}\;#3}

Seeing how easily we've defined type reconstruction for simple types,
one may want to go further and attempt type reconstruction for System-F.
However let's recall inference rule for type application and answer important question,
how does one substitute for unification variable?

\begin{mathpar}
  \inferrule{\Delta;\Gamma\vdash e : \forall\alpha\ldotp\tau}
            {\Delta;\Gamma\vdash e\;* : \tau\{\tau'/\alpha\}}
\end{mathpar}

For example how would we approach this unification:

\begin{mathpar}
  \UVarX\{\tau/\alpha\} \unify \tau_1\to\tau_2
\end{mathpar}

Seems like we have two options. We can create two new variables $\UVarY$, $\UVarZ$
and substitution $[\UVarX \mapsto \UVarY \to \UVarZ]$ or assume that $\UVarX = \alpha$.
That gives us two new paths to consider of which neither is a clear favourite.

\begin{mathpar}
	\UVarY\{\tau/\alpha\}\unify\tau_1\text{ and }
	\UVarZ\{\tau/\alpha\}\unify\tau_2\and
	\tau \unify \tau_1 \to\tau_2
\end{mathpar}

This along with other unclear difficulties results in this problem exploding
and ultimately makes it undecidable.

So let's backtrack. Problem started with substitution of unification variable.
What if we can ensure that $\alpha$ never occurs in $\UVarX$, then $\UVarX\{\tau/\alpha\}=\UVarX$.
First instead of polymorphic types we will use type schemes with monomorphic types:

\begin{alignat*}{2}
  \tau   & \Coloneqq \alpha \mid \tau \to \tau \mid \UVarX \mid \dots
    \tag{types} \\
  \sigma & \Coloneqq \forall\alpha\ldotp\sigma \mid \tau
    \tag{type schemes}
\end{alignat*}

Then inference rules:

\begin{mathpar}
  % variable
	\inferrule{(x : \forall \alpha_1\ldots\alpha_n\ldotp \tau) \in \Gamma}
	  {\Gamma\vdash x \Uparrow
    	\tau\{\UVarX_1/\alpha_1,\ldots,\UVarX_n/\alpha_n\} /
    	\varnothing}

  % lambda
  \inferrule{\Gamma, x : \UVarX \vdash e \Uparrow \tau_2 / \theta}
            {\Gamma\vdash \lambda x.e \Uparrow \theta^*(\UVarX) \to \tau_2 / \theta}

  % application - arrow
  \inferrule{\Gamma\vdash e_1 \Uparrow \tau_2 \to \tau_1 / \theta_1 \\ \theta_1^* \Gamma\vdash e_2 \Downarrow \tau_2 / \theta_2}
            {\Gamma\vdash e_1\;e_2 \Uparrow \tau_1 / \theta_2 \circ \theta_1}

  \\\\

  % application - unification variable
  \inferrule{\Gamma\vdash e_1 \Uparrow \UVarX / \theta_1 \\ \theta_1'^*\Gamma\vdash e_2 \Downarrow \UVarY / \theta_2}
            {\Gamma\vdash e_1\;e_2 \Uparrow \theta_2^*(\UVarZ) / \theta_2 \circ \theta_1'}

  \text{where }\theta_1' = \theta_1[\UVarX \mapsto \UVarY \to \UVarZ]

  % top-down to bottom-up
  \inferrule{\Gamma\vdash e \Uparrow \tau' / \theta \\ \theta^*\tau \unify \tau' / \theta'}
            {\Gamma\vdash e \Downarrow \tau / \theta' \circ \theta}

  \\\\
\end{mathpar}

Let rule (declarative):

\[
  \{\alpha_1, \dots, \alpha_n\} = \FVars\tau \setminus \FVars\Gamma
\]

\begin{mathpar}
  \inferrule{\Gamma \vdash e_1 \colon \tau \\
               \Gamma, x \colon \forall\alpha_1, \dots, \alpha_n \ldotp \tau
                 \vdash e_2 \colon \tau'}
            {\Gamma \vdash \Let{x}{e_1}{e_2} \colon \tau'}

\end{mathpar}

Let rule (algorithmic):

\begin{mathpar}
  \inferrule{\Gamma \vdash e_1 \Uparrow \tau/\theta \\
               \theta^*\Gamma, x\colon\forall\alpha_1, \dots, \alpha_n \ldotp
                 \tau \{ \alpha_1/\UVarX_1, \dots, \alpha_n/\UVarX_n\}
                 \vdash e_2 \Uparrow \tau'/\theta'}
            {\Gamma \vdash \Let{x}{e_1}{e_2} \Uparrow \tau'/\theta'\circ\theta}

  \text{where } \{\UVarX_1, \dots, \UVarX_n\}
    = \FUVars{\tau}\setminus\FUVars{\theta^*\Gamma}
\end{mathpar}

Now, since every unification variable that is being substituted was created in $e_1$,
if it remained unknown it is truely a free variable so can be polymorphic,
and every unification variable that is not being substituted cannot contain $\alpha$,
since at the time of it's creation, $\alpha$ wasn't in the scope,therefore $\UVarX\{\tau/\alpha\}=\UVarX$ remains true.

%%%%%%%%%%%%%%%%%%%%%%%%%%%%%%%%%%%%%%%%%%%%%%%%%%%%%%%%%%%%%%%%%%%%%%%%%%%%%%%
\section{Records and Row-Polymorphism}

\begin{alignat*}{2}
  \tau   & \Coloneqq \alpha \mid \tau \to \tau \mid \{\rho\} \mid \UVarX \\
  \sigma & \Coloneqq \forall\alpha\ldotp\sigma \mid \forall\alpha^R\ldotp\sigma \mid \tau \\
  \rho   & \Coloneqq \cdot \mid l : \tau, \rho \mid \UVarR \mid \alpha^R \\
  e      & \Coloneqq \ldots \mid e.l \mid \{ l = e, \ldots, l = e \} \\
  v      & \Coloneqq \ldots \mid \{ l = v, \ldots, l = v \}^v \\
  E      & \Coloneqq \ldots \mid E.l \mid
    \{ l_1 = v_1, \ldots, l_{i-1} = v_{i-1}, l_i = E, l_{i+1} = e_{i+1}, \ldots, l_n = e_n \}
\end{alignat*}

Typing rules for record types:

\begin{mathpar}
  \inferrule{\Delta; \Gamma \vdash e : \{ l_1 : \tau_1, \ldots, l_n : \tau_n \}}
            {\Delta; \Gamma \vdash e.l_i : \tau_i}

  \inferrule{\Delta; \Gamma \vdash e_1 : \tau_1 \\ \ldots \\ \Delta; \Gamma \vdash e_n : \tau_n}
            {\Delta; \Gamma \vdash \{ l_1 = e_1, \ldots, l_n = e_n \} : \{ l_1 : \tau_1, \ldots, l_n : \tau_n \}}

  \inferrule{\Delta; \Gamma \vdash e : \tau \\ \tau \equiv \tau'}
            {\Delta; \Gamma \vdash e : \tau'}

\end{mathpar}

Congruence rules for types:

\begin{mathpar}
  \inferrule{ }
            {\tau \equiv \tau}

  \inferrule{\tau_1 \equiv \tau_1' \\ \tau_2 \equiv \tau_2'}
            {\tau_1 \to \tau_2 \equiv \tau_1' \to \tau_2'}

  \inferrule{l_1 \not= l_2}
            {l_1 : \tau_1, l_2 : \tau_2, \rho \equiv l_2 : \tau_2, l_1 : \tau_1, \rho}

  \\\\

  \inferrule{ }
            {\cdot \equiv \cdot}

  \inferrule{\rho_1 \equiv \rho_2}
            {\{\rho_1\} \equiv \{\rho_2\}}

  \inferrule{\tau_1 \equiv \tau_2 \\ \rho_1 \equiv \rho_2}
            {l : \tau_1, \rho_1 \equiv l : \tau_2, \rho_2}
\end{mathpar}

Inference rules:

\begin{mathpar}
  \inferrule{\Delta; \Gamma \vdash e \Uparrow \{\rho\} / \theta \\
             \rho \setminus (l :\;\UVarY) \leadsto \rho' / \theta'}
            {\Delta; \Gamma \vdash e.l \Uparrow \theta^*(\UVarY) / \theta'\circ\theta}

  \inferrule{\rho \setminus (l_2 : \tau_2) \leadsto \rho' / \theta \\ l_1 \not= l_2}
            {l_1 : \tau_1, \rho \setminus (l_2 : \tau_2) \leadsto (l_1 : \theta^*\tau_1), \rho' / \theta}

  \inferrule{\UVarR \not\in \FUVars{\tau}}
            {\UVarR \setminus (l : \tau) \leadsto \UVarR' / [\UVarR \mapsto l : \tau, \UVarR']}

  \inferrule{\tau_1 \unify \tau_2 / \theta}
            {l : \tau_1, \rho \setminus (l : \tau_2) \leadsto \theta^*\rho / \theta}
\end{mathpar}

Unification rules:

\begin{mathpar}
  \inferrule{ }
            {\UVarR \unify \UVarR}

  \inferrule{\UVarR \not\in \FUVars{\rho}}
            {\UVarR \unify \rho / [\UVarR \mapsto \rho]}

  \inferrule{ }
            {\cdot \unify \cdot}

  \inferrule{\rho' \setminus (l : \tau) \leadsto \rho'' \\ \theta^*\rho \unify \rho'' / \theta'}
            {l : \tau, \rho \unify \rho' / \theta'\circ\theta}
\end{mathpar}

%%%%%%%%%%%%%%%%%%%%%%%%%%%%%%%%%%%%%%%%%%%%%%%%%%%%%%%%%%%%%%%%%%%%%%%%%%%%%%%
\section{Further Reading}
