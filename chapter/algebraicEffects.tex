\chapter{Algebraic effects}

%%%%%%%%%%%%%%%%%%%%%%%%%%%%%%%%%%%%%%%%%%%%%%%%%%%%%%%%%%%%%%%%%%%%%%%%%%%%%%%
\section{Motivation}

% Pythagorean triples with backtracking in Helium
\begin{verbatim}
  effect BT = {
    fail : unit => a
    filp : unit => Bool
  }

  let rec select a b =
    if a >= b then fail ()
    else if flip () then a
    else select (a + 1) b

  let triples n =
    let a = select 1 n in
    let b = select 1 n in
    let c = select 1 n in
    if a*a + b*b = c*c then (a,b,c)
    else fail ()

  let allTriples n =
    handle triples n with
    | fail () => []
    | flip () => resume True @ resume False
    | return x => [x]
    end
\end{verbatim}

%%%%%%%%%%%%%%%%%%%%%%%%%%%%%%%%%%%%%%%%%%%%%%%%%%%%%%%%%%%%%%%%%%%%%%%%%%%%%%%
\section{Syntax and semantics}

\newcommand\Do[2]{\texttt{do}_{#1}\;#2}
\newcommand\HandleWith[3]{\texttt{handle}_{#1}\;#2\;\texttt{with}\;#3}
\newcommand\Handler[5]{#1,#2\Rightarrow#3\;\texttt{return}\;#4\Rightarrow#5}
\newcommand\Effect[2]{\texttt{effect}\;#1\;\texttt{in}\;#2}

\begin{alignat*}{2}
  e & \Coloneqq v \mid e\;e \mid \Do{l}{e} \mid \HandleWith{l}{e}{h} \mid \Effect{l}{e} \\
  v & \Coloneqq x \mid \lambda x.e \mid () \\
  h & \Coloneqq \Handler{x}{k}{e}{x}{e} \\
  E & \Coloneqq \square \mid E\;e \mid v\;E \mid \HandleWith{l}{E}{h} \mid \Do{l}{E}
\end{alignat*}

\begin{alignat*}{2}
  E[\Effect{l}{e}] & \longrightarrow E[e] \quad (l \text{ -- fresh in } E) \\
  (\lambda x.e) & \rightharpoonup e\{v/x\} \\
  \HandleWith{l}{v}{h} & \rightharpoonup e_2\{v/x\} \\
  \HandleWith{l}{E[\Do{l}{v}]}{h} & \rightharpoonup e_1\{v/x, \lambda y.\HandleWith{l}{E[y]}{h}/k\} \quad (l \not\in sp(E)) \\
  \text{where } \\ & h = \Handler{x}{k}{e_1}{x}{e_2}
\end{alignat*}

%%%%%%%%%%%%%%%%%%%%%%%%%%%%%%%%%%%%%%%%%%%%%%%%%%%%%%%%%%%%%%%%%%%%%%%%%%%%%%%
\section{Type-and-effect system}

\begin{alignat*}{2}
  \tau & \Coloneqq 1 \mid \tau \to_\varepsilon \tau \\
  \varepsilon & \Coloneqq \emptyset \mid l \mid \varepsilon\cdot\varepsilon
\end{alignat*}

Effect row subtyping:

\begin{mathpar}
  \inferrule{}
            {\emptyset <: \varepsilon}

  \inferrule{}
            {\varepsilon <: \varepsilon}

  \inferrule{}
            {\varepsilon <: \varepsilon \cdot l}

  \inferrule{\varepsilon_1 <: \varepsilon_2 \\ \varepsilon_2 <: \varepsilon_3}
            {\varepsilon_1 <: \varepsilon_3}
  \\\\
  \inferrule{\varepsilon_1 <: \varepsilon \\ \varepsilon_2 <: \varepsilon}
            {\varepsilon_1\cdot\varepsilon_2 <: \varepsilon}

  \inferrule{\varepsilon <: \varepsilon_1 \\ \varepsilon <: \varepsilon_2}
            {\varepsilon <: \varepsilon_1\cdot\varepsilon_2}
\end{mathpar}

Function subtyping:

\begin{mathpar}
  \inferrule{\tau_2 <: \tau_1 \\ \tau_1' <: \tau_2' \\ \varepsilon_1 <: \varepsilon_2}
            {\tau_1\to_{\varepsilon_1}\tau_1' <: \tau_2\to_{\varepsilon_2}\tau_2'}
\end{mathpar}

Type system:

\begin{mathpar}
  \inferrule{(x : \tau) \in \Gamma}
            {\Sigma;\Gamma\vdash x : \tau / \emptyset}

  \inferrule{\Sigma;\Gamma, x : \tau_1 \vdash e : \tau_2 / \varepsilon}
            {\Sigma;\Gamma\vdash \lambda x.e : \tau_1 \to_\varepsilon \tau_2 / \emptyset}

  \inferrule{\Sigma;\Gamma\vdash e_1 : \tau_2 \to_\varepsilon \tau_1 / \varepsilon \\ \Sigma;\Gamma\vdash e_2 : \tau_2 / \varepsilon}
            {\Sigma;\Gamma\vdash e_1\;e_2 : \tau_1 / \varepsilon}

  \inferrule{\Sigma;\Gamma\vdash e : \tau / \varepsilon \\ \tau <: \tau' \\ \varepsilon <: \varepsilon'}
            {\Sigma;\Gamma\vdash e : \tau' / \varepsilon'}

  \inferrule{\Sigma, l : \tau_1\Rightarrow\tau_2;\Gamma\vdash e : \tau / \varepsilon \\ l \not\in\varepsilon}
            {\Sigma;\Gamma\vdash \Effect{l}{e} : \tau / \varepsilon}

  \inferrule{(l : \tau_1\Rightarrow\tau_2) \in \Sigma \\ \Sigma;\Gamma\vdash e : \tau_1 / \varepsilon}
            {\Sigma;\Gamma\vdash \Do{l}{e} : \tau_2 / l\cdot\varepsilon}
  \\\\
  \inferrule{(l : \tau_1\Rightarrow\tau_2) \in \Sigma \\ \Sigma;\Gamma, x : \tau \vdash e_3 : \tau_0 / \varepsilon \\\\
             \Sigma;\Gamma\vdash e_1 : \tau / l\cdot\varepsilon \\ \Sigma;\Gamma, x : \tau_1, k : \tau_2 \to_\varepsilon\tau_0 \vdash e_2 : \tau_0 / \varepsilon}
            {\Sigma;\Gamma\vdash \HandleWith{l}{e_1}{\Handler{x}{k}{e_2}{x}{e_3}} : \tau_0 / \varepsilon}
\end{mathpar}

%%%%%%%%%%%%%%%%%%%%%%%%%%%%%%%%%%%%%%%%%%%%%%%%%%%%%%%%%%%%%%%%%%%%%%%%%%%%%%%
\section{Properties}

\begin{theorem}[Progress]
    If $\Sigma;\emptyset \vdash e:\tau/\varepsilon$, then one of the following holds:
    \begin{enumerate}[label=(\roman*)]
        \item $e\longrightarrow e'$
        \item $e = v$ for some $v$
        \item $e = E[\Do{l}{v}]$
            for $l \not\in sp(E)$ and $l\in\varepsilon$
    \end{enumerate}
\end{theorem}

\begin{theorem}[Preservation]
    If $\Sigma;\emptyset \vdash e:\tau/\varepsilon$ and $e\longrightarrow e'$,\\
    then $\Sigma';\emptyset \vdash e':\tau/\varepsilon$
    (for some $\Sigma' \supseteq \Sigma$).
\end{theorem}

%%%%%%%%%%%%%%%%%%%%%%%%%%%%%%%%%%%%%%%%%%%%%%%%%%%%%%%%%%%%%%%%%%%%%%%%%%%%%%%
\section{Polymorphism}

Syntax:
\begin{alignat*}{2}
  e        & \Coloneqq \ldots \mid e\;* \\
  v        & \Coloneqq \ldots \mid \Lambda e \\
  \tau     & \Coloneqq \ldots \mid \forall\alpha.\tau \\
  \epsilon & \Coloneqq \ldots \mid \alpha
\end{alignat*}
(variables are effect variables)

Typing rules:
\begin{mathpar}
  \inferrule{\Delta,\alpha;\Sigma;\Gamma\vdash e : \tau / \emptyset}
            {\Delta;\Sigma;\Gamma\vdash \Lambda e : \forall\alpha.\tau / \emptyset}

  \inferrule{\Delta;\Sigma;\Gamma\vdash e : \forall\alpha.\tau / \varepsilon}
            {\Delta;\Sigma;\Gamma\vdash e\;* : \tau\{\varepsilon'/\alpha\} / \varepsilon}
\end{mathpar}

\begin{verbatim}
  f : (unit -> unit) -> unit

  effect tick in

  count f g =
    handle
      f (\().tick (); g ())
    with
    | tick ()   => \n.resume () (n+1)
    | return () => \n.n
    end 0
\end{verbatim}

Problem: \texttt{f} may handle effect \texttt{tick}.

\begin{mathpar}
  \inferrule{(l : \tau_1\Rightarrow\tau_2) \in \Sigma \\
             \Sigma;\Gamma, x : \tau \vdash e_3 :
               \tau_0 / \varepsilon \setminus l \\\\
             \Sigma;\Gamma\vdash e_1 : \tau / \varepsilon \\
             \Sigma;\Gamma, x : \tau_1,
               k : \tau_2 \to_{\varepsilon\setminus l} \tau_0 \vdash e_2 :
               \tau_0 / \varepsilon\setminus l}
            {\Sigma;\Gamma\vdash
             \HandleWith{l}{e_1}{\Handler{x}{k}{e_2}{x}{e_3}} :
             \tau_0 / \varepsilon}
\end{mathpar}


\begin{alignat*}{2}
  \emptyset \setminus l &= \emptyset \\
  (\varepsilon_1 \cdot \varepsilon_2) \setminus l &=
    (\varepsilon_1 \setminus l) \cdot (\varepsilon_2 \setminus l)\\
  l' \setminus l &= l', \text{ if } l \not= l'\\
  l \setminus l  &= \emptyset
\end{alignat*}

Problem: What is $\alpha \setminus l$?

\[
  \varepsilon  \Coloneqq l \mid \emptyset \mid \varepsilon\cdot\varepsilon
    \mid \alpha \setminus \{ \bar{l} \}
\]

\begin{alignat*}{2}
  \alpha\setminus\{\bar{l}\}\setminus l' &= \alpha\setminus\{l', \bar{l}\}\\
  \Subst{\left(\alpha\setminus\{\bar{l}\}\right)}{\varepsilon}{\alpha} &=
    \varepsilon\setminus\{\bar{l}\}
\end{alignat*}


%%%%%%%%%%%%%%%%%%%%%%%%%%%%%%%%%%%%%%%%%%%%%%%%%%%%%%%%%%%%%%%%%%%%%%%%%%%%%%%
\section{Logical relations}

Binary:

\begin{alignat*}{2}
  \semcatset{Type} & = \mathcal{P}(\syncatset{Value} \times \syncatset{Value}) \\
  \semcatset{Effect} & = \mathcal{P}(\syncatset{Expr}^2 \times \mathcal{P}_{fin}(\syncatset{Label})^2 \times \semcatset{Type})
\end{alignat*}

Unary:

\begin{alignat*}{2}
  \semcatset{Type} & = \mathcal{P}(\syncatset{Value}) \\
  \semcatset{Effect} & = \mathcal{P}(\syncatset{Expr} \times \mathcal{P}_{fin}(\syncatset{Label}) \times \semcatset{Type})
\end{alignat*}

\newcommand\DenotEnv[1]{\Denot{#1}^\theta_\eta}

\begin{alignat*}{2}
  \DenotEnv{\tau_1 \to_\varepsilon \tau_2} & =
    \{v \mid \forall v' \in \DenotEnv{\tau_1}\ldotp v\;v' \in \mathcal{E}\DenotEnv{\tau_2}\DenotEnv{\varepsilon}\} \\
\end{alignat*}

\begin{alignat*}{2}
  \DenotEnv{\emptyset} & = \emptyset \\
  \DenotEnv{\varepsilon_1 \cdot \varepsilon_2} & = \DenotEnv{\varepsilon_1} \cup \DenotEnv{\varepsilon_2} \\
  \DenotEnv{\alpha} & = \eta(\alpha) \\
  \DenotEnv{l} & = \{(\Do{l}{v}, \{l\}, R_2) \mid v \in R_1 \land \theta(l) = (R_1, R_2) \} \\
  \DenotEnv{\varepsilon \setminus l} & = \{(e, l, r) \in \DenotEnv{\varepsilon} \mid l \not\in L\}
\end{alignat*}

\begin{alignat*}{2}
  \mathcal{E}R\,F & = \{e \mid \forall E \in \mathcal{K}R\,F \ldotp E[e] \in \Obs\} \\
  \mathcal{K}R\,F & = \{E \mid (\forall v \in R \ldotp E[v] \in \Obs) \land (\forall e \in \mathcal{S}R\,F \ldotp E[e] \in \Obs)\} \\
  \mathcal{S}R\,F & = \{E[e] \mid (e, L, R_0) \in F \land (\forall l \in L \ldotp l \not\in sp(E)) \land \forall v \in R_0 \ldotp E[v] \in \mathcal{E}R\,F\}
\end{alignat*}

%%%%%%%%%%%%%%%%%%%%%%%%%%%%%%%%%%%%%%%%%%%%%%%%%%%%%%%%%%%%%%%%%%%%%%%%%%%%%%%
\section{Further Reading}
